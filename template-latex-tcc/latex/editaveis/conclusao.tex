\chapter[Conclusão]{Status Atual do Trabalho}\label{chap:conclusao}

Neste Capítulo é apresentado o estado atual deste Trabalho, bem como os resultados obtidos até então. Divide-se em seções,
iniciando com \hyperref[sec:atvstat]{Status das Atividades}, que apresenta o andamento das Atividades que foram especifícadas
na Seção \hyperref[sec:fluxoatv]{Fluxo das Atividades/} \hyperref[sec:fluxoatv]{Subprocessos - Capítulo 4 - Metodologia}. Em seguida, apresenta-se a 
Seção \hyperref[sec:estobjesp]{Status dos} \hyperref[sec:estobjesp]{Objetivos Específicos}, que aborda o andamento dos Objetivos Específicos definidos
no \hyperref[subsec:objesp]{Objetivos Específicos - Capítulo 1} \hyperref[subsec:objesp]{ - Introdução}. Por último, tem-se a Seção de
\hyperref[sec:resumoconc]{Resumo do} \hyperref[sec:resumoconc]{Capítulo} que recapitula o que foi abordado no Capítulo.

\section{Status das Atividades}\label{sec:atvstat}
Esta primeira etapa do trabalho, teve como foco atividades que fundamentam e baseiam a proposta deste trabalho,
pesquisas e construção de conhecimento, bem como testes práticos para experimentar em modelos de Inteligência Articial.
Com base no \hyperref[sec:fluxoatv]{Fluxo das Atividades/Subprocessos - Capítulo 4 - Metodologia}, o Quadro \hyperref[tab:5]{5}
apresenta o estado das Atividades definidas para a primeira etapa do trabalho:

\begin{table}[htbp]
    \centering
    \begin{threeparttable}
        \caption{Andamento das Atividades da Primeira Etapa do TCC}
        \label{tab:5}
        \begin{tabular}{| c | c |}
        \hline  
        Atividades & Andamento \\
        \hline 
        Definir Tema & Concluída \\
        \hline 
        Levantar Bibliográfia & Concluída \\
        \hline 
        Descrever Introdução & Concluída \\
        \hline 
        Levantamento do Referencial Teórico &  Concluída \\
        \hline 
        Descrever Suporte Tecnológico & Concluída \\
        \hline 
        Descrever Metodologia & Concluída \\
        \hline 
        Desenvolvimento Prova de Conceito & Concluída \\
        \hline 
        Descrever Proposta & Concluída \\
        \hline 
        Descrever Conclusões iniciais & Concluída \\
        \hline
        Apresentar Monografia & Não Concluída \\
        \hline 
        \end{tabular}
        \begin{tablenotes}
            \small
            \centering
            \item Fonte: Autora
        \end{tablenotes}
    \end{threeparttable}
\end{table}

Na segunda etapa deste trabalho, o foco estará em desenvolver a Proposta apresentada no 
\hyperref[chap:proposta]{Capítulo 5 - Proposta}, aplicando a teoria apresentada no 
\hyperref[chap:refteor]{Capítulo 2 - Referencial} \hyperref[chap:refteor]{Teórico}, com o suporte das tecnologias
mostradas no \hyperref[chap:suptec]{Capítulo 3 - Referencial Teórico}. O Quadro \hyperref[tab:6]{6}
traz o andamento das atividades referentes a segunda etapa do trabalho:

\begin{table}[htbp]
    \centering
    \begin{threeparttable}
        \caption{Andamento das Atividades da Segunda Etapa do TCC}
        \label{tab:6}
        \begin{tabular}{| c | c |}
        \hline 
        Atividades & Andamento \\
        \hline 
        Aplicar Correções da Banca & Não Concluída \\
        \hline 
        Separar Base de Dados & Não Concluída \\
        \hline 
        Desenvolvimento Modelo de Sistema de Recomendação & Não Concluída \\
        \hline 
        Desenvolvimento da API & Não Concluída \\
        \hline 
        Aplicar Modelo na API & Não Concluída \\
        \hline 
        Analisar Resultados & Não Concluída \\
        \hline 
        Usar os resultados para incrementar o modelo & Não Concluída \\
        \hline 
        Reaplicar novo modelo na API &  Não Concluída \\
        \hline 
        Análise dos novos resultados &  Não Concluída \\
        \hline 
        Refinar Monografia &  Não Concluída \\
        \hline
        Apresentar Monografia &  Não Concluída \\
        \hline 
        \end{tabular}
        \begin{tablenotes}
            \small
            \centering
            \item Fonte: Autora
        \end{tablenotes}
    \end{threeparttable}
\end{table}

\section{Status dos Objetivos Específicos}\label{sec:estobjesp}
Na Seção \hyperref[subsec:objesp]{Objetivos Específicos - Capítulo 1 - Introdução}, foram definidos os Objetivos
Específicos deste trabalho, nos quais alguns já foram concluídos nesta primeira etapa. Sendo estes:

\begin{itemize}
    \item Levantamento sobre Sistemas de Recomendação: foi apresentado no 
    \hyperref[chap:refteor]{Capítulo 2 - } \hyperref[chap:refteor]{Referencial Teórico};
    \item Levantamento, no contexto da IA, de algoritmos de \textit{Deep Learning} que mais atendam à proposta desse trabalho:
    foi apresentado no \hyperref[chap:refteor]{Capítulo 2 - Referencial Teórico} e no 
    \hyperref[chap:proposta]{Capítulo 5 - Proposta};
    \item Treinamento da base de dados orientando-se pelos algoritmos de \textit{Deep Learning} levantados:
    foi apresentado no \hyperref[chap:proposta]{Capítulo 5 - Proposta};
\end{itemize}

\section{Resumo do Capítulo}\label{sec:resumoconc}
Na primeira etapa deste trabalho, foi levantado a bibligráfia necessária para embasar teoricamente este trabalho, e também
foi realizado algumas provas de conceito para testar a teoria e definir o melhor modelo a ser utilizado no trabalho. Dessa forma,
todas as atividades previstas para a primeira parte do trabalho, com exceção da apresentação deste, foram concluídas. 
Conluiu-se também alguns dos \hyperref[subsec:objesp]{Objetivos Específicos} definidos no \hyperref[chap:intro]{Capítulo 1 - Introdução}.
Já para a segunda etapa, será realizado o desenvolvimento do que foi proposto na primeira etapa e concluir os 
\hyperref[subsec:objesp]{Objetivos Específicos} e o \hyperref[subsec:objgeral]{Objetivo Geral}.