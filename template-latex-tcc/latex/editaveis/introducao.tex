\chapter[Introdução]{Introdução}

Este capítulo introduz o contexto, no qual esta trabalho está inserido. Inicialmente, aborda-se a 
\hyperref[sec:contextualizacao]{Contextualização}, que traz o foco de interesse desse trabalho (inteligência 
artificial e sistemas de recomendação). Na sequência, consta a \hyperref[sec:justificativa]{Justificativa}
para a realização do mesmo, além da \hyperref[sec:questaopesquisa]{Questões de Pesquisa e Desenvolvimento} a ser respondida, e os
\hyperref[sec:objetivos]{Objetivos} que se pretendem ser alcançados. E por fim, é abordado a 
\hyperref[sec:organizacao]{Organização da Monografia}. 

\section{Contextualização}\label{sec:contextualizacao}
Sistemas de recomendação (SR) são algoritmos de software projetados para analisar o comportamento passado do usuário 
e suas preferências, com o intuito de sugerir itens que possam ser de seu interesse \cite{Subramaniam}. No contexto
da sociedade atual, com grande conjunto de dados disponíveis, tem se tornado cada vez mais desafiador para os usuários encontrarem 
exatamente o que procuram em meio às vastas opções. Dessa forma, os sistemas de recomendação surgem como
meios para facilitar a descoberta de conteúdo relevante em ambientes digitais. Grandes empresas utilizam 
sistemas de recomendação como estratégia para conquistar e auxiliar seus clientes, como o Youtube e a Netflix \cite{Zhang_Survey}.

Os benefícios dos sistemas de recomendação na sociedade atual são significativos. Eles não apenas facilitam a descoberta
de novos produtos e conteúdos, mas também aumentam o engajamento do usuário e impulsionam as vendas e a receita para
empresas e plataformas digitais. Além disso, ao personalizar a experiência do usuário, os sistemas de recomendação podem
melhorar a satisfação do cliente e a fidelidade à marca \cite{Gunawardana2022}.

No entanto, os sistemas de recomendação possuem desafios. De acordo com a pesquisa Khusro; Ali; Ullah (2016)  
os principais problemas que ocorrem no contexto dos sistemas de recomendação são:
\begin{itemize}
\item \textit{Cold Start}: ocorre quando há nenhum ou poucos dados para iniciar as sugestões, acontecendo principalmente 
com novos usuários. Esse desafio também é conhecido como "partida a frio";
\item Esparcidade dos dados: ocorre quando a matriz de dados dos usuários possui  muitos espaçõs vazios. Isso ocorre, principalmente,
devido à grande quantidade de itens, o que dificulta o preenchimento pleno desses itens; ou ainda pela falta de avaliações
dos usuários para cada item;
\item Escalabilidade: ocorre por haver a necessidade de uma grande rede de usuários, gerando outros problemas, por exemplo,
baixo desempenho;
\item Diversidade: ocorre pela dificuldade em conciliar itens diferentes (diferenciação) e itens similares (sobreposição). A
diferenciação permite aos usuários receberem itens mais variados, algo que pode ser do seu interesse. Já a sobreposição confere itens
apenas itens similares, não recomendando itens que possuem ligações mais distantes, e
\item Efeito de habituação: ocorre quando o grande número de informações gera um efeito de costume no usuário, ou seja,
certa rotina em suas preferências. Isso pode fazer com que várias recomendações sejam ignoradas, recomendando-se sempre a mesma
coisa.
\end{itemize}

Outra tecnologia que vem avançando hoje é a Inteligência Artificial (IA), que tem permeado diversos setores da nossa sociedade \cite{perspectiva-dados-IA-2023}. IA refere-se à capacidade de sistemas computacionais executarem tarefas que 
normalmente exigiriam inteligência humana, tais como: aprendizado, raciocínio, reconhecimento de padrões e tomada de decisões. 
Uma área específica da IA que tem ganhado destaque é o \textit{Deep Learning} (DL), um ramo do aprendizado de 
máquina que se baseia em redes neurais artificiais para realizar tarefas complexas de forma automatizada.
Essas tecnologias trazem vários benefícios, desde a automação de processos até a analise de grandes volumes de dados, 
otimizando operações e impulsionando a tomada de decisões em diversos setores (saúde, transporte, entretenimento, dentre outros)
\cite{Suleimenov}.

No intuito de mitigar os desafios inerentes aos sistemas de recomendação, pode-se fazer uso de \textit{Deep Learning}.

Usar Deep Learning pode ajudar a mitigar o desafio Cold Start, ou "partida a frio". Isso ocorre, pois Deep Learning 
permite aprender representações complexas a partir de dados brutos. Ao utilizar técnicas como redes neurais, os sistemas 
de recomendação podem extrair padrões e relações significativas dos dados, mesmo quando há poucos dados disponíveis para
novos usuários. Isso permite que o sistema forneça recomendações mais precisas e relevantes, mesmo para usuários com 
histórico limitado \cite{expressanalytics-cold-start-problem}.

No desafio de esparcidade dos dados, Deep Learning permite aprender com base em representações densas de informações, 
mesmo que a matriz que comporta tais informações não seja plena, com todos os itens preenchidos, restando espaços em branco
\cite{HEIDARI2022109835}. Nesse cenário, técnicas como autoencoders podem ser usadas para preencher lacunas nos 
dados, gerando representações completas e semanticamente coerentes. Explicações complementares de como funcionam tais 
técnicas serão conferidas no Capítulo 2 - Referencial Teórico.

Por serem altamente paralelizáveis, técnicas de Deep Learning podem ser implementadas em sistemas distribuídos. 
Essa estratégia permite lidar com grande quantidade de dados e usuários. Ao usar técnicas de treinamento distribuído e 
otimização eficiente, os sistemas de recomendação baseados em Deep Learning podem escalar para grandes conjuntos de dados
e redes de usuários sem comprometer o desempenho \cite{10.1145/2783258.2783270}.

Deep Learning ajuda a melhorar a diversidade das recomendações ao aprender representações
mais abrangentes dos itens e usuários. Além disso, permite que o sistema de recomendação perceba similaridades com 
base nos itens e usuários. Como resultado, tende-se a recomendações mais equilibradas e acertivas. Nos estudos realizados 
até o momento pela autora, cabe menção à técnica de introdução de fatores de diversidade no processo de geração de 
recomendações, conforme abordado em \cite{kingma2016improving}. Essa técnica busca garantir que os usuários recebam uma variedade de 
itens relevantes. Constam outros detalhes sobre o funcionamento dessa técnica no Capítulo 2 - Referencial Teórico.

Sistemas de recomendação orientados a Deep Learning são capazes de aprender continuamente, centrados nos \textit{feedbacks}
dos usuários. Adicionalmente, são sistemas capazes de adaptar as recomendações ao longo do tempo.
Essas particularidades de aprendizado contínuo e adaptabilidade possibilitam que as recomendações permaneçam atualizadas
e relevantes aos interesses dos usuários, mesmo após longos períodos de uso \cite{lomonaco2019continual}. Portanto, isso mitiga o efeito de
habituação comentado anteriormente nesse capítulo.

Diante do exposto, usar recursos da IA em sistemas de recomendação representa uma oportunidade para melhorar as 
recomendações conferidas aos usuários. Ao combinar algoritmos avançados de IA
com dados detalhados sobre o comportamento do usuário, é possível gerar recomendações
mais precisas e relevantes, que atendam às necessidades e preferências individuais do usuário de forma mais eficaz
\cite{Karatzoglou}. Essa abordagem híbrida permite que os sistemas de recomendação aprendam e se adaptem continuamente, 
melhorando sua capacidade de prever e antecipar as preferências do usuário ao longo do tempo.

\section{Justificativa}\label{sec:justificativa}
Nos últimos anos, tem sido observado um aumento na relevância da Web como meio para transações 
eletrônicas e comerciais, que é um fator importante no desenvolvimento de sistemas de recomendação centrados em Inteligência 
Artificial (IA) \cite{Aggarwal2016}. Isso se dá principalmente pelas vantagens que esse tipo de sistema
confere às companhias, tais como: aumento do número de vendas; maior diversidade de itens
vendidos; aumento da satisfação dos usuários, maior fidelidade desses usuários, e melhor atendimento quanto às necessidades 
dos usuários \cite{Gunawardana2022}. Dentre as companhias de comércio eletrônico que já fazem uso de sistemas de 
recomendação centrados na IA, destacam-se: Amazon, Netflix, IMDb e Youtube. As mesmas utilizam sistemas de recomendação 
para oferecer sugestões personalizadas aos seus usuários \cite{deconstructing-recommender-systems}.

Apesar dos benefícios proporcionados pelos sistemas de recomendação, também é importante reconhecer suas limitações
e desafios. Estudos têm demonstrado que muitos usuários não fornecem \textit{feedbacks} ou avaliações aos sistemas de 
recomendação, o que resulta em um problema conhecido como "partida a frio" (\textit{cold start}), ou ainda dificultam na
escalabilidade, à medida que a rede de usuários cresce ou a base de dados necessita de atualização \cite{Mishra_2021}.  

Em meio a esse cenário, torna-se relevante considerar o impacto da crescente utilização da IA no
desenvolvimento e e na evolução de sistemas de recomendação. Estudos indicam que o número de empresas que utilizam 
serviços baseados em IA cresceu significativamente nos últimos anos, o que evidencia a importância e o potencial dessas
tecnologias \cite{perspectiva-dados-IA-2023}. É nesse contexto que se destaca a relevância da aplicação de técnicas 
avançadas de IA, como redes neurais, para aprimorar a capacidade dos sistemas de recomendação em entender
e antecipar as preferências dos usuários \cite{Zhang_Survey}.
 
Considerando os beneficios apontados anteriormente no que compreende o uso de IA em sistemas de recomendação, o presente trabalho 
visa a realização de um estudo, procurando investigar estratégias e técnicas avançadas de IA para o desenvolvimento
de sistemas de recomendação mais eficientes e precisos. Sistemas esses capazes de proporcionar uma experiência 
personalizada e satisfatória aos usuários em diferentes contextos e aplicações.

\section{Questões de Pesquisa e Desenvolvimento}\label{sec:questaopesquisa}
Ao final desse trabalho, pretende-se responder às seguintes questões:

Questão de Desenvolvimento: Como um sistema de recomendação centrado em Inteligência Artificial pode ser desenvolvido? 

Questão de Pesquisa: Esse sistema de recomendação centrado em IA, de fato, proporciona maior satisfação aos seus usuários?

Com a exposição das questões, cabem dois esclarecimentos adicionais.

O primeiro compreende deixar claro sobre o escopo do sistema de recomendação pretendido. Pretende-se que seja um sistema 
de recomendação que faz uso de algumas técnicas de IA, sendo essas as mais recomendadas pela literatura em estudo. 
Tais técnicas serão escolhidas com base nos desafios acordados anteriormente, orientando-se pelos autores 
Khusro; Ali; Ullah (2016). Será um sistema de escopo menor, mas que será documentado para permitir que outros interessados 
consigam usá-lo de base.

Segundo Monique Esperidião (2005), satisfação de usuário é um critério qualitativo. Portanto, trata-se de um requisito não
funcional, subjetivo. No intuito de conferir maior clareza sobre como "mensurar" a satisfação do usuário, pretende-se 
fazer uso de questionários, junto ao público alvo de cada domínio de interesse utilizado ao longo desse trabalho, 
para coletar \textit{feedbacks} sobre as impressões desses usuários quanto às recomendações recebidas, sem e com o auxílio de um 
sistema de recomendação centrado em IA. A decisão de usar um questionário baseou-se na literatura. Segundo a autora 
Monique Esperidião (2005), dentre os métodos quantitativos, destacam-se \textit{surveys}, incluindo o uso de questionários 
fechados. Nesses questionários, as questõoes estão associadas a uma escala de valores que procuram mensurar a satisfação 
dos usuários. Na verdade, constumam mensurar aspectos associados à satisfação, no caso: expectativas e percepções. 
Ainda de acordo com Monique Esperidião (2005), há possibilidade de usar métodos qualitativos, mas deve-se ter em mente 
que os mesmos costumam ser criticados pelos especialistas, por incorrerem em pesquisas com víes, dentre outros problemas. 
O presente trabalho fará uso de uma abordagem híbrida, quantitativa e qualitativa, conforme alguns autores defendem 
\cite{minayo1993}. Nesse contexto, cabe colocar que, nessa abordagem híbrida, a parte quantitativa e a parte qualitativa 
não se encontram em oposição, mas sim de continuidade e complementaridade. O questionário se orientará por uma escala em 
números, ou seja, quantitativa, mas que seguirá uma análise interpretativa sobre esses números, sendo assim conferindo, 
adicionalmente, um viés qualitativo. Colocacões complementares sobre o método de análise de resultados ocorrerão no 
Capítulo 4 - Metodologia.

\section{Objetivos}\label{sec:objetivos}
Os objetivos do trabalho foram organizados em Objetivo Geral, que confere uma visão mais abrangente do que se pretende 
atingir com a realização do trabalho, e Objetivos Específicos, compreendendo metas de menor escopo a serem cumpridas 
para que seja viável atingir o objetivo geral.

\subsection{Objetivo Geral}
Esse trabalho tem como objetivo geral um estudo sobre sistemas de recomendação centrados em Inteligência Artificial, 
revelando como desenvolvê-los e analisando se os mesmos melhoram a satisfação dos usuários.

\subsection{Objetivos Específicos}
Na perspectiva de alcançar o objetivo geral, foram definidos objetivos específicos, conforme apresentado a seguir:

\begin{enumerate}
    \item Levantamento sobre sistemas de recomendação;
    \item Levantamento, no contexto da IA, de algoritmos de Deep Learning que mais atendam à proposta desse trabalho;
    \item Identificação de quais informações serão necessárias para especificar uma base de dados para o projeto;
    \item Treinamento da base de dados orientando-se pelos algoritmos de Deep Learning levantados;
    \item Desenvolvimento de um sistema de recomendação centrado em IA, mais especificamente usando algoritmos de Deep Learning, e
    \item Análise da satisfação do usuário sem e com o uso do sistema de recomendação desenvolvido.
\end{enumerate}

\section{Organização da Monografia}\label{sec:organizacao}

Esta monografia está organizada em capítulos, conforme consta a seguir:

\textbf{Capítulo 2 - Referencial Teórico:} aborda fundamentos, conceitos, princípios e práticas
relevantes para o tema, em específico, sistemas de recomendação centrados em Inteligência Artificial;

\textbf{Capítulo 3 - Suporte Tecnológico:} apresenta as principais tecnologias utilizadas para
o desenvolvimento do trabalho;

\textbf{Capítulo 4 - Metodologia:} esclarece sobre os métodos que guiam a elaboração do trabalho, apresentando desde a classificação da 
pesquisa, e detalhando os métodos investigativo, de desenvolvimento e de análise de resultados;

\textbf{Capítulo 5 - Proposta:} confere uma visão mais detalhada da proposta, procurando retomar o contexto, e apresentando outras 
nuances sobre os algoritmos de Deep Learning escolhidos e a preparação da base de dados, e

\textbf{Capítulo 6 - Conclusão:} evidencia o status atual do trabalho, retomando objetivos já alcançados nessa primeira 
etapa, em andamento, e que ainda serão cumpridos com a realização da segunda etapa. Além disso, procura-se destacar 
sobre as principais contribuições do trabalho e as percepções da autora ao se concluir a primeira etapa. 
