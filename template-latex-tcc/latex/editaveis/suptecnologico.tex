\chapter[Suporte Tecnológico]{Suporte Tecnológico}

Este Capítulo descreve as ferramentas e tecnologias utilizadas na elaboração deste trabalho, sendo separadas
em seções com base no tipo de apoio provido. As ferramentas que apóiam a pesquisa dos tópicos de interesse desse 
trabalho estão na seção
\hyperref[sec:apoiopesquisa]{Apoio à Pesquisa}. Já as ferramentas que conferem apoio mais técnico, destacando linguagem
de programação e plataforma de desenvolvimento, estão na seção de \hyperref[sec:apoiodev]{Apoio ao Desenvolvimento}.
As ferramentas que auxiliam na escrita deste trabalho são apresentadas na seção de \hyperref[sec:elaboracaomono]
{Apoio à Elaboração da Monografia}. Por fim, tem-se o \hyperref[sec:resumosuptec]
{Resumo do Capítulo}.

\section{Apoio à Pesquisa}\label{sec:apoiopesquisa}
Essa seção é voltada para as ferramentas utilizadas no apoio à pesquisa de modelos de Inteligência Artificial, bem
como à exemplificação de bases de dados.

\subsection{Kaggle}\label{subsec:kaggle}
O Kaggle é uma comunidade \textit{online} de cientistas de dados e praticantes de aprendizado de máquina, que 
oferece acesso a conjuntos de dados diversificados e desafios competitivos que impulsionam a inovação. Possue 
uma grande diversidade de conjuntos de dados e modelos disponíveis, os quais os pesquisadores podem explorar e 
experimentar para aprimorar seus modelos de dados \cite{kagglesite}. Será utilizada no trabalho no intuito de explorar
modelos de Inteligência Artificial já aplicados na comunidade, servindo de apoio na construção do modelo desse trabalho.
Além de apresentar conjuntos de dados que serão utilizados para a montagem da base de dados usada para treinamento do modelo 
a ser criado. Foi escolhida essa comunidade, pela grande quantidade de base de dados disponíveis e modelos pré treinados que 
podem servir de base para o desenvolvimento. 

\subsection{HuggingFace}\label{subsec:huggingface}
O HuggingFace oferece uma biblioteca de modelos pré-treinados, bem como uma plataforma para compartilhamento e 
colaboração na criação de modelos de linguagem natural. Com sua vasta coleção de modelos, os pesquisadores têm à 
disposição recursos vantajosos para aprimorar a eficiência e a precisão dos seus modelos \cite{huggingsite}. No presente trabalho será 
utilizada visando apoiar o aprendizado durante o treinamento do modelo a ser desenvolvido, tendo a disposição vários modelos
pré-treinados que servirão de inspiração e possíveis referências.

\section{Apoio ao desenvolvimento}\label{sec:apoiodev}
Essa seção foca em ferramentas utilizadas no processo de desenvolvimento prático desse trabalho.

\subsection{Python}\label{subsec:python}
O Python é uma das linguagens de programação mais populares no campo da Inteligência Artificial e da ciência de dados, a 
qual oferece bibliotecas e \textit{frameworks} especializados, tais como TensorFlow, PyTorch e Scikit-learn. Esses recursos
facilitam o 
desenvolvimento e a implementação de algoritmos de recomendação. Sua sintaxe clara e expressiva, além de sua flexibilidade 
tornam a linguagem Python uma das melhores opções para projetos com Aprendizado de Máquina \cite{pythonsite}. Dessa forma,
será a linguagem utilizada nesse trabalho para desenvolver o modelo de Inteligência Artificial.

\subsection{Colab/GPU}\label{subsec:colab}
O Google Colab é uma plataforma baseada em nuvem que permite o desenvolvimento de código Python de forma colaborativa 
e totalmente gratuita. Além disso, oferece acesso a recursos de GPU diretamente do Google, sem precisar consumir 
\textit{hardware} local. Considerando que o treinamento de modelos demanda muita memória, ter esse acesso torna mais 
simples e eficiente ao treinar os modelos durante o desenvolvimento \cite{colabsite}. Por isso, essa plataforma será utilizada
para o desenvolvimento do modelo em conjunto com a linguagem Python.

\subsection{Google \textit{Forms} e Google \textit{Sheets}}\label{subsec:forms}
O Google \textit{Forms} é uma plataforma gratuita de formulários online, que possibilita a criação de questionários, pesquisas, formulários, 
entre outros. A ferramenta suporta uma variedade de tipos de perguntas, oferecendo flexibilidade na criação de questionários.
Além de ser de fácil acesso, através de qualquer dispositivo conectado à internet. E também possuir integração com o Google 
\textit{Sheets}, levando as respostas dos formulários para planilhas, o que facilita a visualização e a análise dos dados. 
Por isso, para validar a satisfação dos usuários após o desenvolvimento da aplicação será criado questionários no Google 
\textit{Forms} e para melhor visualização dos dados, serão transferidos para o Google \textit{Sheets} para análise.

\section{Apoio à Elaboração da Monografia}\label{sec:elaboracaomono}
Essa seção apresenta as ferramentas utilizadas para redigir esse trabalho.

\subsection{LaTeX}\label{subsec:latex}
O LaTeX é um sistema de preparação de documentos que se concentra na estruturação lógica do conteúdo, permitindo
uma formatação consistente e padronizada do trabalho \cite{latexsite}. O uso do LaTeX se deu por sua versatilidade no quesito
de formatar os documentos, utilizando um modelo já pré estabelecido pela instituição de ensino.

\subsection{Git/GitHub}\label{subsec:git}
O Git e o GitHub são ferramentas para controle de versionamento, sendo possível acompanhar as alterações no texto,
facilitar a revisão e armazenar o projeto como um todo, garantindo a consistência e a integridade do trabalho \cite{githubsite}.

\subsection{Canva}\label{subsec:canva}
O Canva \cite{canva} é um site que permite a criação e a edição de images, sendo possível o \textit{download} em diferentes formatos.
Dessa forma, foi a ferramenta utilizada na elaboração e na edição de imagens desse trabalho. O uso dessa ferramenta se deu porque
em sua versão gratuita confere várias ferramentas de criação de edição de imagem, além de difentes formatos de baixar as imagens.

\subsection{Lucidchart}\label{subsec:lucidchart}
O Lucidchart é uma plataforma web de diagramação e visualização de informações, com diversos modelos pré-definidos para 
diferentes tipos de diagramas, como fluxogramas, organogramas, entre outros \cite{lucidchart}. Sua versão gratuita disponibiliza não só vários
modelos de diagramas, como também ferramentas e formas. Por essa facilidade, a plataforma será usada no trabalho para
a criação de diagramas.

\section{Resumo do Capítulo}\label{sec:resumosuptec}
Neste Capítulo, foram descritas as tecnologias que estão auxiiliando a viabilização deste deste trabalho. Foram apresentadas as 
ferramentas que apóiam a pesquisa, com menção ao Kaggle e ao HuggingFace. 
Como ferramentas de apoio ao desenvolvimento, destacou-se a linguagem Python e a plataforma Colab. 
Por fim, foram apresentadas as ferramentas que auxiliam na elaboração dessa monografia, com foco em LaTeX, Git/GitHub e Canva.

O Quadro \hyperref[tab:2]{2} apresenta as ferramentas utilizadas nas suas respectivas versões, com uma breve descrição e a documentação ou site 
oficial.

\begin{table}[htbp]
    \centering
    \begin{threeparttable}
        \caption{Principais ferramentas utilizadas no trabalho}
        \label{tab:2}
        \begin{tabular}{>{\centering\arraybackslash}m{3cm} c m{5cm} c}
        \toprule 
        Ferramenta & Versão & Descrição & Documentação \\
        \midrule
        Kaggle & - & Comunidade de cientistas de dados voltado para IA & \cite{kagglesite} \\
        \hline 
        HuggingFace & - & Comunidade de cientistas de dados voltado para IA & \cite{huggingsite} \\
        \hline 
        Python & 3 & Linguagem de programação de alto nível & \cite{pythonsite} \\
        \hline 
        Colab & - & Plataforma de hospedagem de código & \cite{colabsite} \\
        \hline 
        LaTeX & LaTeX 2e2 & Linguagem de marcação & \cite{latexsite} \\
        \hline 
        Git & 2.44.0 & Sistema de versionamento de dados & \cite{gitsite} \\
        \hline 
        GitHub & - & Sistema de versionamento de dados & \cite{githubsite} \\
        \hline 
        Canva & - & Edição de imagem & \cite{canva} \\
        \hline 
        Lucidchart & - & Plataforma de diagramação & \cite{lucidchart} \\
        \hline 
        Google \textit{Forms} & - & Ferramenta de formulários  & \cite{forms} \\
        \hline 
        Google \textit{Sheets} & - & Ferramenta de planilhas  & \cite{sheets} \\
        \bottomrule 
        \end{tabular}
        \begin{tablenotes}
            \small
            \centering
            \item Fonte: Autora
        \end{tablenotes}
    \end{threeparttable}
\end{table}
