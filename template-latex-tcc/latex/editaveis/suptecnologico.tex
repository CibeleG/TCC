\chapter[Suporte Tecnológico]{Suporte Tecnológico}

Este Capítulo traz as ferramentas e tecnologias utilizadas de suporte na elaboração deste trabalho. Separadas
em seções com base no tipo de apoio provido. As ferramentas que apoiaram a pesquisa e serviram de recursos estão na seção
\hyperref[sec:apoiopesquisa]{Apoio à pesquisa}. Já as ferramentas que trouxeram apoio mais técnico, destacando linguagem
de programação e plataforma de desenvolvimento estão na seção de \hyperref[sec:apoiodev]{Apoio ao desenvolvimento}.
E por fim, as ferramentas que auxiliaram na escrita deste trabalho se encontram na seção de \hyperref[sec:elaboracaomono]
{Elaboração da Monografia}.

\section{Apoio à pesquisa}\label{sec:apoiopesquisa}
Essa seção é voltada para as ferramentas utilizadas no apoio da pesquisa de modelos de Inteligência Artificial, bem
como exemplificação de bases de dados.

\subsection{Kaggle}\label{subsec:kaggle}
O Kaggle é uma comunidade online de cientistas de dados e praticantes de aprendizado de máquina, que 
oferece acesso a conjuntos de dados diversificados e desafios competitivos que impulsionam a inovação. Possue 
uma grande diversidade de conjuntos de dados e modelos disponíveis, os quais os pesquisadores podem explorar e 
experimentar para aprimorar seus modelos de dados \cite{kagglesite}.

\subsection{HuggingFace}\label{subsec:huggingface}
O HuggingFace oferece uma biblioteca de modelos pré-treinados, bem como uma plataforma para compartilhamento e 
colaboração na criação de modelos de linguagem natural. Com sua vasta coleção de modelos, os pesquisadores têm à 
disposição recursos poderosos para aprimorar a eficiência e a precisão dos seus modelos \cite{huggingsite}.

\section{Apoio ao desenvolvimento}\label{sec:apoiodev}
Essa seção traz as ferramentas utilizadas no processo de desenvolvimento prático dessa monografia.

\subsection{Python}\label{subsec:python}
O Python é uma das linguagens de programação mais populares no campo da Inteligência Artificial e ciência de dados, a 
qual oferece bibliotecas e frameworks especializados, como TensorFlow, PyTorch e Scikit-learn, que facilitam o 
desenvolvimento e a implementação de algoritmos de recomendação. Sua sintaxe clara e expressiva, além da sua flexibilidade 
o tornam uma das melhores opções para projetos com Aprendizado de Máquina \cite{pythonsite}.

\subsection{Colab/GPU}\label{subsec:colab}
O Google Colab é uma plataforma baseada em nuvem que permite o desenvolvimento de código Python de forma colaborativa 
e totalmente gratuita. Além disso, oferece acesso a recursos de GPU diretamente do Google, sem precisar consumir 
\textit{hardware} local, considerando que o treinamento de modelos demanda muita memória, ter esse acesso torna mais 
simples e eficiente ao treinar os modelos durante o desenvolvimento \cite{colabsite}.

\section{Elaboração da Monografia}\label{sec:elaboracaomono}
Essa seção apresenta as ferramentas utilizadas para redigir esse trabalho.

\subsection{LaTeX}\label{subsec:latex}
O LaTeX é um sistema de preparação de documentos que se concentra na estruturação lógica do conteúdo, permitindo
uma formatação consistente e padronizada do trabalho \cite{latexsite}.

\subsection{Git/GitHub}\label{subsec:git}
O Git e o GitHub são ferramentas para controle de versionamento, sendo possível acompanhar as alterações no texto,
facilitar a revisão e armazenar o projeto como um todo, garantindo a consistência e a integridade do trabalho \cite{githubsite}.

\section{Resumo do Capítulo}\label{sec:resumosuptec}
Este Capítulo tem como objetivo descrever as tecnologias que auxiliaram no desenvolvimento deste trabalho. Foi apresentado
as ferramentas que apoiaram a pesquisa em Apoio à pesquisa, como Kaggle e HuggingFace. Ainda as ferramentas no Apoio ao
desenvolvimento como Python e Colab. E por fim, foi apresentado ferramentas que auxiliaram na Elaboração da Monografia, tais
como LaTeX e Git/GitHub.

A Tabela \hyperref[tab:1]{1} apresenta as ferramentas utilizadas nas suas respectivas versões, com uma breve descrição e a documentação ou site 
oficial.

\begin{table}[htbp]
    \centering
    \begin{threeparttable}
        \caption{Principais ferramentas utilizadas no trabalho}
        \label{tab:1}
        \begin{tabular}{>{\centering\arraybackslash}m{3cm} c m{5cm} c}
        \toprule 
        Ferramenta & Versão & Descrição & Documentação \\
        \midrule
        Kaggle & - & Comunidade de cientistas de dados voltado para IA & \cite{kagglesite} \\
        \hline 
        HuggingFace & - & Comunidade de cientistas de dados voltado para IA & \cite{huggingsite} \\
        \hline 
        Python & 3 & Linguagem de programação de alto nível & \cite{pythonsite} \\
        \hline 
        Colab & - & Plataforma de hospedagem de código & \cite{colabsite} \\
        \hline 
        LaTeX & LaTeX 2e2 & Linguagem de marcação & \cite{latexsite} \\
        \hline 
        Git & 2.44.0 & Sistema de versionamento de dados & \cite{gitsite} \\
        \hline 
        GitHub & - & Sistema de versionamento de dados & \cite{githubsite} \\
        \bottomrule 
        \end{tabular}
        \begin{tablenotes}
            \small
            \centering
            \item Fonte: Autora
        \end{tablenotes}
    \end{threeparttable}
\end{table}
