\chapter[Suporte Tecnológico]{Suporte Tecnológico}

\section{Apoio à pesquisa}\label{sec:apoiopesquisa}
Essa seção é voltada para as ferramentas utilizadas no apoio da pesquisa de modelos de Inteligência Artificial, bem
como exemplificação de bases de dados.

\subsection{Kaggle}\label{subsec:kaggle}
O Kaggle é uma comunidade online de cientistas de dados e praticantes de aprendizado de máquina, que 
oferece acesso a conjuntos de dados diversificados e desafios competitivos que impulsionam a inovação. Possue 
uma grande diversidade de conjuntos de dados e modelos disponíveis, os quais os pesquisadores podem explorar e 
experimentar para aprimorar seus modelos de dados \cite{kagglesite}.

\subsection{HuggingFace}\label{subsec:huggingface}
O HuggingFace oferece uma biblioteca de modelos pré-treinados, bem como uma plataforma para compartilhamento e 
colaboração na criação de modelos de linguagem natural. Com sua vasta coleção de modelos, os pesquisadores têm à 
disposição recursos poderosos para aprimorar a eficiência e a precisão dos seus modelos \cite{huggingsite}.

\section{Apoio ao desenvolvimento}\label{sec:apoiodev}
Essa seção traz as ferramentas utilizadas no processo de desenvolvimento prático dessa monografia.

\subsection{Python}\label{subsec:python}
O Python é uma das linguagens de programação mais populares no campo da Inteligência Artificial e ciência de dados, a 
qual oferece bibliotecas e frameworks especializados, como TensorFlow, PyTorch e Scikit-learn, que facilitam o 
desenvolvimento e a implementação de algoritmos de recomendação. Sua sintaxe clara e expressiva, além da sua flexibilidade 
o tornam uma das melhores opções para projetos com Aprendizado de Máquina \cite{pythonsite}.

\subsection{Colab/GPU}\label{subsec:colab}
O Google Colab é uma plataforma baseada em nuvem que permite o desenvolvimento de código Python de forma colaborativa 
e totalmente gratuita. Além disso, oferece acesso a recursos de GPU diretamente do Google, sem precisar consumir 
\textit{hardware} local, considerando que o treinamento de modelos demanda muita memória, ter esse acesso torna mais 
simples e eficiente ao treinar os modelos durante o desenvolvimento \cite{colabsite}.

\section{Elaboração da monografia}\label{sec:elaboracaomono}
Essa seção apresenta as ferramentas utilizadas para redigir esse trabalho.

\subsection{LaTeX}\label{subsec:latex}
O LaTeX é um sistema de preparação de documentos que se concentra na estruturação lógica do conteúdo, permitindo
uma formatação consistente e padronizada do trabalho \cite{latexsite}.

\subsection{Git/GitHub}\label{subsec:git}
O Git e o GitHub são ferramentas para controle de versionamento, sendo possível acompanhar as alterações no texto,
facilitar a revisão e armazenar o projeto como um todo, garantindo a consistência e a integridade do trabalho \cite{githubsite}.

\section{Resumo do Capítulo}\label{sec:resumosuptec}
A Tabela 1 apresenta as ferramentas utilizadas nas suas respectivas versões, com uma breve descrição e a documentação ou site 
oficial.

\begin{table}
    \centering
    \caption{Fonte: Autora}
    \label{tab:1}
    \begin{tabular}{|c|c|c|}
    \hline
    Ferramenta & Versão & Descrição & Documentação \\
    \hline
    Kaggle & Dado 1.2 & Dado 1.3 & \cite{kagglesite}\\
    HuggingFace & Dado 2.2 & Dado 2.3 & \cite{huggingsite}\\
    Python & Dado 3.2 & Dado 3.3 & \cite{pythonsite}\\
    Colab & Dado 3.2 & Dado 3.3 & \cite{colabsite}\\
    LaTeX & Dado 3.2 & Dado 3.3 & \cite{latexsite}\\
    Git & Dado 3.2 & Dado 3.3 & \cite{gitsite}\\
    GitHub & Dado 3.2 & Dado 3.3 & \cite{githubsite}\\
    \hline
    \end{tabular}
\end{table}
