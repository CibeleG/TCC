\chapter[Referencial Teórico]{Referencial Teórico}
Este Capítulo apresenta as bases teóricas que suportam esse trabalho. Estando organizado em seções, sendo estas:
Seção \hyperref[sec:sisrec]{2.1}, que apresenta Sistemas de Recomendação; Seção \hyperref[sec:ia]{2.2}, a qual traz 
o conceito de Inteligência Artificial e recursos desta que serão utilizados neste trabalho; Seção \hyperref[sec:expus]{2.3},
que aborda a experiência do usuário e sua importância para este trabalho. E por último, são apresentadas as considerações finais
do capítulo.

\section{Sistemas de Recomendação}\label{sec:sisrec}

Os Sistemas de Recomendação são aplicações de software que analisa e processa dados dos usuários com o proposito de 
sugerir itens que possam ser de interesse desses usuários. Assim, eles ajudam os usuários a descobrir novos produtos, 
conteúdos ou serviços que possam ser do seu interesse, personalizando suas experiências \cite{pham2019recommendation}.

\begin{figure}[h]
    \centering
    \includegraphics[width=0.5\textwidth]{figuras/ciclosr.eps}
    \caption{Ciclo dos Sistemas Recomendação}
    \label{fig:ciclosr}
    \small Fonte: Autora
\end{figure}

O processo de um sistema de recomendação pode ser dividido em várias etapas, como demonstrado na figura 
\hyperref[fig:ciclosr]{1}. O ciclo começa na coleta de dados, nesta etapa o sistema coleta dados relevantes para analise
do perfil do usuário, como interações com o sistema, \textit{feedbacks}, fontes externas como midias sociais também podem ser
dados válidos. Como o (Kurt and Murali, 2016) exemplificam, o Spotify coleta informações dos artistas, sinais acústicos das músicas,
histórico e em tempo real músicas ouvidas.
Após a coleta, temos o armazenamento e filtragem desses dados, em geral para popular a base de dados, quanto maior a quantidade de dados,
melhor para a análise e separar esses dados para melhorar a acurácia do modelo \cite{pham2019recommendation}. Em seguida, 
temos a análise desses dados, no caso desse trabalho, utilizando algoritmos de \textit{deep learning} para detectar padrões,
e encontrar as preferências do usuário. Dessa forma, conseguindo gerar recomendações em potencial que serão sugeriadas para o usuário,
entrando na última parte do ciclo, que é a avaliação pelo usuário da recomendação, e com base nessa avaliação, o modelo 
de recomendação recebe mais dados do perfil daquele usuário e o ciclo recomeça.

\section{Inteligência Artificial}\label{sec:ia}

Inteligência Artificial é um conjunto de tecnologias que permitem aos computadores executar uma variedade de 
funções avançadas, incluindo a capacidade de ver, entender e traduzir idiomas falados e escritos, analisar dados, 
fazer recomendações e muito mais \cite{Suleimenov}.

No contexto de Sistemas de Recomendação, é feito o uso principalmente de quatro técnicas da Inteligência Artificial 
\cite{stratoflow-recommendation}:
\begin{itemize}
\item Filtros Colaborativos: essa técnica foca na similaridade entre diferentes usuários e itens. Usuários que são
similares provavelemente tem o mesmo tipo de interesse \cite{pham2019recommendation}, como demonstrado na Figura \hyperref[fig:filtrocolab]{2};

\begin{figure}[htbp]
    \centering
    \includegraphics[width=0.5\textwidth]{figuras/filtrocolab.eps}
    \caption{Filtro Colaborativo}
    \label{fig:filtrocolab}
    \small Fonte: Autora
\end{figure}

\item Filtros de Conteúdo: essa técnica avalia a similaridade entre os itens, itens similares ao que o usuário gostou/visualizou
tem mais probabilidade de ser de seu interesse \cite{stratoflow-recommendation}, demonstrado na Figura \hyperref[fig:filtrocont]{3};

\begin{figure}[htbp]
    \centering
    \includegraphics[width=0.5\textwidth]{figuras/filtrocontent.eps}
    \caption{Filtro de Conteúdo}
    \label{fig:filtrocont}
    \small Fonte: Autora
\end{figure}

\item Filtros Híbridos: essa técnica combina as técnicas de Filtro Colaborativo e de Filtro de Conteúdo. Podendo, predizer
cada filtro separadamente e depois combiná-los, ou ainda unificá-los em um único modelo. Esta técnica consegue lidar com as
limitações de ambas as técnicas que ela combina, podendo lidar com usuários de gostos mais diferentes e com situações as quais
tem poucos dados de interação do usuário, assim também lidando com o problema de "partida à frio"\cite{stratoflow-recommendation}, e 

\item Filtros baseados em \textit{Deep Learning}: essa técnica utiliza redes neurais para fazer predições e recomendações, 
como Redes Neurais Convolucionais (RNC) para dados mais voltados para imagens e Redes Neurais Recorrentes (RNR) para dados
sequênciais \cite{nvidia-recommendation}.
\end{itemize}

\textcolor{red}{Pensar em qual modelo usar, se usar filtros hibridos seção abaixo para eles e explicar melhor RNR e RNC.
Se usar deep learning, seção abaixo para explicar melhor os modelos e como usar.}

\subsection{Modelo a ser usado}\label{subsec:modeloaserusado}

\section{Experiência do usuário}\label{sec:expus}
% A experiência do usuário pode ser definida como as respostas e percepções de uma pessoa ao usar um produto, sistema ou serviço
% \cite{iso9241-210}. Mas também podemos dizer que a experiência de usuário vai além de avaliar a usabilidade/funcionalidade durante
% a interação do usuário, ela tende a cobrir não só os comportamentos do sistema, bem como a eficiência e efetividade \cite{allam2013user}
% No contexto de Sistemas de Recomendação, a experiência do usuário 

% \cite{8673410}

\section{Resumo do Capítulo}\label{sec:resrefteor}
