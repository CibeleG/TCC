\begin{resumo}
    Sistemas de recomendação são algoritmos de software projetados para analisar o comportamento e as 
    preferências passadas dos usuários, com o objetivo de sugerir itens que possam ser de seu interesse. 
    Esses sistemas são amplamente utilizados em diversas plataformas, como serviços de streaming, lojas online, 
    redes sociais e outros aplicativos, para personalizar a experiência do usuário e melhorar a satisfação ao oferecer 
    recomendações relevantes e personalizadas. Porém ainda temos muito a aprender em relação à esses Sistemas e como
    eles poderiam gerar mais satiasfação aos seus usuários. Este trabalho visa criar um Sistema de Recomendação com
    aplicação de Inteligência Artificial e avaliar a satisfação do usuário após o uso do Sistema. Para isso, será
    criado um modelo treinado por Filtros Colaborativos, e em seguida será usado esse modelo para treinar um modelo de Rede
    Neural Convolucional, a fim de aprimorar as recomendações. Ao final, será utilizados métricas, como Precisão; Revocação; 
    Erro Médio Absoluto e Erro Quadrático Médio da Raiz, para avaliar o modelo resultante, e ainda será aplicado esse modelo
    a uma API para que os usuários consigam usar e avaliar o Sistema de Recomendação.

 \vspace{\onelineskip}
    
 \noindent
 \textbf{Palavras-chave}: sistemas de recomendação, inteligência artificial, filtros colaborativos, rnc.
\end{resumo}
